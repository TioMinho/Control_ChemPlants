%-------------------------------------------------------
%-- PREAMBLE
%-------------------------------------------------------
\documentclass[8pt]{beamer}
\usetheme[]{Feather}
  
\setbeamersize{text margin left=0.75cm,text margin right=0.75cm}

% INCLUDE PACKAGES
%-------------------------------------------------------

\usepackage[utf8]{inputenc}
\usepackage[english]{babel}
\usepackage[T1]{fontenc}
\usepackage{helvet}

\usepackage{graphicx}
\graphicspath{ {imgs/} }

\usepackage[utf8]{inputenc}
\usepackage[english]{babel}
\usepackage[protrusion=true,expansion=true]{microtype} 
\usepackage{amsmath,amsfonts,amsthm,amssymb,bm}
\usepackage{color, xcolor}
\usepackage{listings}
\usepackage[document]{ragged2e}
\usepackage{wrapfig}
\usepackage{mdframed}
\usepackage{multicol}
\usepackage{environ}

\lstset{
    backgroundcolor=\color[rgb]{0.86,0.88,0.93},
    language=matlab, keywordstyle=\color[rgb]{0,0,1},
    basicstyle=\footnotesize \ttfamily,breaklines=true,
    escapeinside={\%*}{*)}
}

% DEFFINING COLORS
%-------------------------------------------------------
\definecolor{glgRed}{RGB}{214,45,32}
\definecolor{glgGreen}{RGB}{0,135,68}
\definecolor{glgBlue}{RGB}{0,87,231}
\definecolor{glgOrange}{RGB}{255,167,0}

\definecolor{retroBrown}{RGB}{102,101,71}
\definecolor{retroRed}{RGB}{251,46,1}
\definecolor{retroGreen}{RGB}{111,203,159}
\definecolor{retroYellow}{RGB}{255,226,138}

\definecolor{pastelBlue}{RGB}{27,133,184}
\definecolor{pastelRed}{RGB}{174,90,65}
\definecolor{pastelGreen}{RGB}{85,158,131}
\definecolor{pastelBlack}{RGB}{90,82,85}

\definecolor{niceBlack}{RGB}{14,17,17}
\definecolor{niceBlue}{RGB}{14,104,206}

% Beamer Color Scheme
\setbeamercolor{Feather}{fg=niceBlack!20,bg=niceBlue!80}
\setbeamercolor{structure}{fg=niceBlack}
\setbeamercolor{frametitle}{bg=niceBlue!70}
\setbeamercolor{normal text}{fg=black!85}


% DEFFINING AND REDEFINING COMMANDS
%-------------------------------------------------------

% colored hyperlinks
\newcommand{\chref}[2]{
  \href{#1}{{\usebeamercolor[bg]{Feather}#2}}
}

\makeatletter
\let\beamer@writeslidentry@miniframeson=\beamer@writeslidentry
\def\beamer@writeslidentry@miniframesoff{%
  \expandafter\beamer@ifempty\expandafter{\beamer@framestartpage}{}% does not happen normally
  {%else
    % removed \addtocontents commands
    \clearpage\beamer@notesactions%
  }
}
\newcommand*{\miniframeson}{\let\beamer@writeslidentry=\beamer@writeslidentry@miniframeson}
\newcommand*{\miniframesoff}{\let\beamer@writeslidentry=\beamer@writeslidentry@miniframesoff}
\makeatother

\makeatletter
\patchcmd{\beamer@sectionintoc}
  {\vfill}
  {\vskip\itemsep}
  {}
  {}
\makeatother  

% tensor 2:
\newcommand{\tend}[1]{\hbox{\oalign{$\bm{#1}$\crcr\hidewidth$\scriptscriptstyle\bm{\sim}$\hidewidth}}}
\newcommand{\tenq}[1]{\hbox{\oalign{$\bm{#1}$\crcr\hidewidth$\scriptscriptstyle\bm{\sim}$\hidewidth}}}


% ENVIROMENTS
%-------------------------------------------------------

\AtBeginSection[]{
  \begin{frame}
  \vfill
    \centering
    \Huge \color{glgBlue!70} \insertsectionhead\par%
  \vfill
  \end{frame}
}

%\AtBeginSubsection[]{
%  \begin{frame}
%  \vfill
%    \centering
%    \Huge \color{glgBlue!60} \insertsubsectionhead\par%
%    \huge \color{niceBlack!85} \insertsectionhead\par%
%  \vfill
%  \end{frame}
%}

\newenvironment<>{varblock}[2][.9\textwidth]{%
  \setlength{\textwidth}{#1}
  \begin{actionenv}#3%
    \def\insertblocktitle{#2}%
    \par%
    \usebeamertemplate{block begin}}
  {\par%
    \usebeamertemplate{block end}%
  \end{actionenv}}

\NewEnviron{myequation}{%
    \begin{equation*}
    \scalebox{1.1}{$\BODY$}
    \end{equation*}
    }
    
% INFORMATION IN THE TITLE PAGE
%-------------------------------------------------------

\title[Optimal Control: An application to a non-isothermal continuous reactor]
{
      Optimal Control:\\An application to a non-isothermal continuous reactor
}

\subtitle[]
{
}

\author[Otacílio Bezerra Leite Neto]
{      
	Otacílio Bezerra Leite Neto\\
}

\institute[]
{
	TI0153 - Trabalho de Conclusão de Curso II\\
    Department of Teleinformatics Engineering\\
    Federal University of Ceará - UFC\\
  
  %there must be an empty line above this line - otherwise some unwanted space is added between the university and the country (I do not know why ;( )
}

\date{\today}

%-------------------------------------------------------
%-- THE BODY OF THE PRESENTATION
%-------------------------------------------------------

\begin{document}

%-------------------------------------------------------
% THE TITLEPAGE
%-------------------------------------------------------

{ \1
\begin{frame}[plain,noframenumbering] % the plain option removes the header from the title page, noframenumbering removes the numbering of this frame only
  \titlepage % call the title page information from above
\end{frame}}

%-------------------------------------------------------
% SUMMARY
%-------------------------------------------------------

\begin{frame}
	\begin{center} \Large \bfseries
		Summary
	\end{center}
    \begin{columns}[t]
        \begin{column}{0.5\textwidth}
            \tableofcontents[sections={1-3}]
        \end{column}
        \begin{column}{0.5\textwidth}
            \tableofcontents[sections={4-6}]
        \end{column}
    \end{columns}
\end{frame}


%-------------------------------------------------------
% SECTION - Introduction
%-------------------------------------------------------
\section{Introduction}
%-------------------------------------------------------
\subsection{Contextualization}
%-------------------------------------------------------
\begin{frame}[t]{Introduction}{Contextualization}

\vskip0.25cm
{\justifying During the semester, we have discussed Conservation Laws of fluid elements...}


\begin{varblock}[1\linewidth]{Conservation of Mass}
	\begin{equation*}
		{\begin{pmatrix}
			\text{Time rate of} \\
			\text{change of mass} \\
			\text{in the system}
		\end{pmatrix}} = 
		{\begin{pmatrix}
			\text{Mass} \\
			\text{entering} \\
			\text{the system}
		\end{pmatrix}} - 
		{\begin{pmatrix}
			\text{Mass} \\
			\text{leaving} \\
			\text{the system}
		\end{pmatrix}}
	\end{equation*} \vskip0.2cm
\end{varblock}

\end{frame}

%-------------------------------------------------------
\subsection{Problem Definition}
%-------------------------------------------------------
\begin{frame}[t]{Introduction}{Problem Definition}

\vskip0.25cm
{\justifying During the semester, we have discussed Conservation Laws of fluid elements...}


\begin{varblock}[1\linewidth]{Conservation of Mass}
	\begin{equation*}
		{\begin{pmatrix}
			\text{Time rate of} \\
			\text{change of mass} \\
			\text{in the system}
		\end{pmatrix}} = 
		{\begin{pmatrix}
			\text{Mass} \\
			\text{entering} \\
			\text{the system}
		\end{pmatrix}} - 
		{\begin{pmatrix}
			\text{Mass} \\
			\text{leaving} \\
			\text{the system}
		\end{pmatrix}}
	\end{equation*} \vskip0.2cm
\end{varblock}

\end{frame}

%-------------------------------------------------------
% SECTION - Dynamical System Analysis
%-------------------------------------------------------
\section{Dynamical System Analysis}
%-------------------------------------------------------
\subsection{Mathematical Models for Chemical Reactors}
%-------------------------------------------------------
\begin{frame}[t]{Dynamical System Analysis}{Mathematical Models for Chemical Reactors}

\vskip0.25cm
{\justifying During the semester, we have discussed Conservation Laws of fluid elements...}


\begin{varblock}[1\linewidth]{Conservation of Mass}
	\begin{equation*}
		{\begin{pmatrix}
			\text{Time rate of} \\
			\text{change of mass} \\
			\text{in the system}
		\end{pmatrix}} = 
		{\begin{pmatrix}
			\text{Mass} \\
			\text{entering} \\
			\text{the system}
		\end{pmatrix}} - 
		{\begin{pmatrix}
			\text{Mass} \\
			\text{leaving} \\
			\text{the system}
		\end{pmatrix}}
	\end{equation*} \vskip0.2cm
\end{varblock}

\end{frame}

%-------------------------------------------------------
\subsection{General Properties of Dynamical Models}
%-------------------------------------------------------
\begin{frame}[t]{Dynamical System Analysis}{General Properties of Dynamical Models}

\vskip0.25cm
{\justifying During the semester, we have discussed Conservation Laws of fluid elements...}


\begin{varblock}[1\linewidth]{Conservation of Mass}
	\begin{equation*}
		{\begin{pmatrix}
			\text{Time rate of} \\
			\text{change of mass} \\
			\text{in the system}
		\end{pmatrix}} = 
		{\begin{pmatrix}
			\text{Mass} \\
			\text{entering} \\
			\text{the system}
		\end{pmatrix}} - 
		{\begin{pmatrix}
			\text{Mass} \\
			\text{leaving} \\
			\text{the system}
		\end{pmatrix}}
	\end{equation*} \vskip0.2cm
\end{varblock}

\end{frame}

%-------------------------------------------------------
\subsection{Experiments}
%-------------------------------------------------------
\begin{frame}[t]{Dynamical System Analysis}{Experiments}

\vskip0.25cm
{\justifying During the semester, we have discussed Conservation Laws of fluid elements...}


\begin{varblock}[1\linewidth]{Conservation of Mass}
	\begin{equation*}
		{\begin{pmatrix}
			\text{Time rate of} \\
			\text{change of mass} \\
			\text{in the system}
		\end{pmatrix}} = 
		{\begin{pmatrix}
			\text{Mass} \\
			\text{entering} \\
			\text{the system}
		\end{pmatrix}} - 
		{\begin{pmatrix}
			\text{Mass} \\
			\text{leaving} \\
			\text{the system}
		\end{pmatrix}}
	\end{equation*} \vskip0.2cm
\end{varblock}

\end{frame}

%-------------------------------------------------------
% SECTION - State-Feedback Controllers
%-------------------------------------------------------
\section{State-Feedback Controllers}
%-------------------------------------------------------
\subsection{Definitions}
%-------------------------------------------------------
\begin{frame}[t]{State-Feedback Controllers}{Definitions}

\vskip0.25cm
{\justifying During the semester, we have discussed Conservation Laws of fluid elements...}


\begin{varblock}[1\linewidth]{Conservation of Mass}
	\begin{equation*}
		{\begin{pmatrix}
			\text{Time rate of} \\
			\text{change of mass} \\
			\text{in the system}
		\end{pmatrix}} = 
		{\begin{pmatrix}
			\text{Mass} \\
			\text{entering} \\
			\text{the system}
		\end{pmatrix}} - 
		{\begin{pmatrix}
			\text{Mass} \\
			\text{leaving} \\
			\text{the system}
		\end{pmatrix}}
	\end{equation*} \vskip0.2cm
\end{varblock}

\end{frame}

%-------------------------------------------------------
\subsection{Regulation vs. Tracking}
%-------------------------------------------------------
\begin{frame}[t]{State-Feedback Controllers}{Regulation vs. Tracking}

\vskip0.25cm
{\justifying During the semester, we have discussed Conservation Laws of fluid elements...}


\begin{varblock}[1\linewidth]{Conservation of Mass}
	\begin{equation*}
		{\begin{pmatrix}
			\text{Time rate of} \\
			\text{change of mass} \\
			\text{in the system}
		\end{pmatrix}} = 
		{\begin{pmatrix}
			\text{Mass} \\
			\text{entering} \\
			\text{the system}
		\end{pmatrix}} - 
		{\begin{pmatrix}
			\text{Mass} \\
			\text{leaving} \\
			\text{the system}
		\end{pmatrix}}
	\end{equation*} \vskip0.2cm
\end{varblock}

\end{frame}

%-------------------------------------------------------
% SECTION - Optimal Control
%-------------------------------------------------------
\section{Optimal Control}
%-------------------------------------------------------
\subsection{Formulation}
%-------------------------------------------------------
\begin{frame}[t]{Optimal Control}{Formulation}

\vskip0.25cm
{\justifying During the semester, we have discussed Conservation Laws of fluid elements...}


\begin{varblock}[1\linewidth]{Conservation of Mass}
	\begin{equation*}
		{\begin{pmatrix}
			\text{Time rate of} \\
			\text{change of mass} \\
			\text{in the system}
		\end{pmatrix}} = 
		{\begin{pmatrix}
			\text{Mass} \\
			\text{entering} \\
			\text{the system}
		\end{pmatrix}} - 
		{\begin{pmatrix}
			\text{Mass} \\
			\text{leaving} \\
			\text{the system}
		\end{pmatrix}}
	\end{equation*} \vskip0.2cm
\end{varblock}

\end{frame}

%-------------------------------------------------------
\subsection{Linear Quadratic (LQ) Controllers}
%-------------------------------------------------------
\begin{frame}[t]{Optimal Control}{Linear Quadratic (LQ) Controllers}

\vskip0.25cm
{\justifying During the semester, we have discussed Conservation Laws of fluid elements...}


\begin{varblock}[1\linewidth]{Conservation of Mass}
	\begin{equation*}
		{\begin{pmatrix}
			\text{Time rate of} \\
			\text{change of mass} \\
			\text{in the system}
		\end{pmatrix}} = 
		{\begin{pmatrix}
			\text{Mass} \\
			\text{entering} \\
			\text{the system}
		\end{pmatrix}} - 
		{\begin{pmatrix}
			\text{Mass} \\
			\text{leaving} \\
			\text{the system}
		\end{pmatrix}}
	\end{equation*} \vskip0.2cm
\end{varblock}

\end{frame}

%-------------------------------------------------------
\subsection{Simulations}
%-------------------------------------------------------
\begin{frame}[t]{Optimal Control}{Simulations}

\vskip0.25cm
{\justifying During the semester, we have discussed Conservation Laws of fluid elements...}


\begin{varblock}[1\linewidth]{Conservation of Mass}
	\begin{equation*}
		{\begin{pmatrix}
			\text{Time rate of} \\
			\text{change of mass} \\
			\text{in the system}
		\end{pmatrix}} = 
		{\begin{pmatrix}
			\text{Mass} \\
			\text{entering} \\
			\text{the system}
		\end{pmatrix}} - 
		{\begin{pmatrix}
			\text{Mass} \\
			\text{leaving} \\
			\text{the system}
		\end{pmatrix}}
	\end{equation*} \vskip0.2cm
\end{varblock}

\end{frame}

%-------------------------------------------------------
% SECTION - Optimal State Estimation
%-------------------------------------------------------
\section{Optimal State Estimation}
%-------------------------------------------------------
\subsection{Formulation}
%-------------------------------------------------------
\begin{frame}[t]{Optimal State Estimation}{Formulation}

\vskip0.25cm
{\justifying During the semester, we have discussed Conservation Laws of fluid elements...}


\begin{varblock}[1\linewidth]{Conservation of Mass}
	\begin{equation*}
		{\begin{pmatrix}
			\text{Time rate of} \\
			\text{change of mass} \\
			\text{in the system}
		\end{pmatrix}} = 
		{\begin{pmatrix}
			\text{Mass} \\
			\text{entering} \\
			\text{the system}
		\end{pmatrix}} - 
		{\begin{pmatrix}
			\text{Mass} \\
			\text{leaving} \\
			\text{the system}
		\end{pmatrix}}
	\end{equation*} \vskip0.2cm
\end{varblock}

\end{frame}

%-------------------------------------------------------
\subsection{Kalman Filter and LQG Controllers}
%-------------------------------------------------------
\begin{frame}[t]{Optimal State Estimation}{Kalman Filter and Linear Quadratic Gaussian (LQG)}

\vskip0.25cm
{\justifying During the semester, we have discussed Conservation Laws of fluid elements...}


\begin{varblock}[1\linewidth]{Conservation of Mass}
	\begin{equation*}
		{\begin{pmatrix}
			\text{Time rate of} \\
			\text{change of mass} \\
			\text{in the system}
		\end{pmatrix}} = 
		{\begin{pmatrix}
			\text{Mass} \\
			\text{entering} \\
			\text{the system}
		\end{pmatrix}} - 
		{\begin{pmatrix}
			\text{Mass} \\
			\text{leaving} \\
			\text{the system}
		\end{pmatrix}}
	\end{equation*} \vskip0.2cm
\end{varblock}

\end{frame}

%-------------------------------------------------------
\subsection{Simulations}
%-------------------------------------------------------
\begin{frame}[t]{Optimal State Estimation}{Simulations}

\vskip0.25cm
{\justifying During the semester, we have discussed Conservation Laws of fluid elements...}


\begin{varblock}[1\linewidth]{Conservation of Mass}
	\begin{equation*}
		{\begin{pmatrix}
			\text{Time rate of} \\
			\text{change of mass} \\
			\text{in the system}
		\end{pmatrix}} = 
		{\begin{pmatrix}
			\text{Mass} \\
			\text{entering} \\
			\text{the system}
		\end{pmatrix}} - 
		{\begin{pmatrix}
			\text{Mass} \\
			\text{leaving} \\
			\text{the system}
		\end{pmatrix}}
	\end{equation*} \vskip0.2cm
\end{varblock}

\end{frame}

%-------------------------------------------------------
% SECTION - Conclusion
%-------------------------------------------------------
\section{Conclusion}
%-------------------------------------------------------
\begin{frame}[t]{Conclusion}{}

\vskip0.25cm
{\justifying During the semester, we have discussed Conservation Laws of fluid elements...}

\begin{varblock}[1\linewidth]{Conservation of Mass}
	\begin{equation*}
		{\begin{pmatrix}
			\text{Time rate of} \\
			\text{change of mass} \\
			\text{in the system}
		\end{pmatrix}} = 
		{\begin{pmatrix}
			\text{Mass} \\
			\text{entering} \\
			\text{the system}
		\end{pmatrix}} - 
		{\begin{pmatrix}
			\text{Mass} \\
			\text{leaving} \\
			\text{the system}
		\end{pmatrix}}
	\end{equation*} \vskip0.2cm
\end{varblock}

\end{frame}


%-------------------------------------------------------
%-- End Page
{\1
\begin{frame}[plain,noframenumbering]
	\center \Huge \textbf{Thank you!}    	  
	  
	Questions?
\end{frame}

%-------------------------------------------------------
\end{document}


%%---------------------Rascunho--------------------------
%\section{Titulo}
%\subsection{Subtitulo}
%%---------------------Rascunho--------------------------
%\begin{frame}{Titulo}{Subtitulo}
%
%\begin{block}{title}
%	Say somethings \alert{new} 
%\end{block}
%
%\begin{itemize}
%	\item TikZ\footnote{TikZ is a package for creating beautiful graphics. Have a look at these \chref{http://www.texample.net/tikz/examples/}{online examples} or the \chref{http://tug.ctan.org/tex-archive/graphics/pgf/base/doc/generic/pgf/pgfmanual.pdf}{pgf user manual}.}
%\end{itemize}
%
%\end{frame}
%%---------------------Rascunho--------------------------

%%%%%%%%%%%%%%%%%%%%%%%%%%%%%%%%%%%
%% Drafts:
%%%%%%%%%%%%%%%%%%%%%%%%%%%%%%%%%%%
%%%%% Figure:
% \begin{figure}[ht]
% 	\centering
% 	\includegraphics[trim={0cm 0cm 0cm 0cm},clip,scale=1]{nameFigure}
% 	\caption{caption of the figure.}
% 	\label{fig:nameFigure}
% \end{figure} \vskip0.25cm
%
%%%%% Equation:
% \begin{equation} \label{eq:nameEquation}
% \begin{split}
%	 X = 1 + 1
% \end{split}
% \end{equation} \vskip0.25cm
%
%%%% Table:
% \begin{table}[hp]
% 	\centering
% 	\begin{tabular}{l | c c }
% 	Principal & Coluna1 & Coluna2 \\
% 	\hline 
% 	ABC	& 1 & 2 \\
% 	DFG	& 3 & 4 \\
% 	HIJ	& 5 & 6 \\
% 	\end{tabular} 
% 	\caption{caption of the table.}
% 	\label{table:nameTable}	
% \end{table} \vskip0.25cm
